\documentclass[letterpaper,12pt,]{article}

\usepackage[%
    left=1in,%
    right=1in,%
    top=1in,%
    bottom=1.0in,%
    paperheight=11in,%
    paperwidth=8.5in%
]{geometry}%

\usepackage{listings}
\usepackage{graphicx}
\usepackage{amsmath}
\usepackage[font=small,skip=-2pt]{caption}
\usepackage{subcaption}
\usepackage{hyperref}
\usepackage{booktabs}
\usepackage{pdfpages}
\usepackage{pgffor}
\usepackage[section]{placeins}


\lstdefinestyle{mystyle}{
    %backgroundcolor=\color{backcolour},
    %commentstyle=\color{codegreen},
    %keywordstyle=\color{magenta},
    %numberstyle=\tiny\color{codegray},
    %stringstyle=\color{codepurple},
    basicstyle=\footnotesize,
    breakatwhitespace=false,
    breaklines=true,
    captionpos=b,
    keepspaces=true,
    numbers=left,
    numberstyle=\footnotesize,
    stepnumber=1,
    numbersep=5pt,
    showspaces=false,
    showstringspaces=false,
    showtabs=false,
    tabsize=2,
    frame=single
}
\lstset{frame=single}

\pagestyle{empty} % Remove page numbering
\linespread{1.5} % Line Spacing

\begin{document}

\input{titlepage.tex}

\section*{Question 1}

Figure \ref{fig:q1} shows a left-facing rarefaction and a right-facing shock. The control surface has moved further into the driven section. The tail of the expansion fan has a positive velocity since it is now located in the driven section.

\begin{figure}[!ht]
    \centering
    \includegraphics[width = 0.95\textwidth]{p1.pdf}
    \caption {Problem 1: Rarefaction-shock}
    \label{fig:q1}
\end{figure}

\section*{Question 2}

\begin{figure}[!ht]
    \centering
    \includegraphics[width = 0.95\textwidth]{p2.pdf}
    \caption {Problem 2: Rarefaction-Rarefaction}
    \label{fig:q2}
\end{figure}

Figure \ref{fig:q2} shows left-facing and right-facing rarefactions.
The contact surface velocity is zero as shown on the velocity plot, and we can confirm it has not moved from its original position $x = x_0$.

\section*{Question 3}

\begin{figure}[!ht]
    \centering
    \includegraphics[width = 0.95\textwidth]{p3.pdf}
    \caption {Problem 3: Rarefaction-shock}
    \label{fig:q3}
\end{figure}

Figure \ref{fig:q3} shows a left-facing rarefaction and a right-facing shock.
This test cases exhibits the same features as in Problem 1.


\section*{Code}

Code has been written in FORTRAN available on my Github

\url{https://github.com/dougshidong/mech539/tree/master/a5}

\end{document}
